To understand energy consumption, execution time and various optimizations,
a light-weight fine-grained measurement tool is required. RCRtool provides 
user-level fast access to hardware counters.
Finding the optimal combination of compiler optimizations 
requires a compilation framework, like the Polyhedral Compiler
Collection, that easily produces a large number of 
program variants with specific optimization parameters.

\subsection{RCRtool on Sandy Bridge System}
The Intel Sandy Bridge architecture allows users to track energy usage through 
the exposed Running Average Power Limit (RAPL) interface\cite{IntelSystemProgrammingVol3}. 
A model-specific register (MSR) was added with the Sandy Bridge to track energy
consumed by the chip  -- MSR\_PKG\_ENERGY\_STATUS.
The counter is frequently updated and counts the energy in 15.3 micro-Joule units.
Experiments have shown\cite{us} and the documentation\cite{IntelSystemProgrammingVol3} states
that the counter can be accessed as often as every microsecond. 
The counter is only 32 bits and can wrap in as little as a couple of minutes.
The RCRtool detects the wraps and supplies a 64 bit value with the 
upper 32 bits being the number of wraps since RCRtool instantiation.
The RCRtool must run at supervisor level to access the counter.
It writes the current value of the counter at least 1000 times a second into a shared-memory
data structure. This ``blackboard'' structure provides a hierarchical view of the system
where various current performance information is stored. The 
information is available to any OpenMP applications through a simple API that 
delineates a code region for measurement with a start and end call.
Each region is identified by its file name and line number.
If a region is executed multiple times the energy is summed across all executions. 
All energy information is available during application shutdown.

The ROSE source-to-source compiler\cite{ROSE} 
finds OpenMP parallel regions and adds RCRtool API calls around the region 
automatically. ROSE tracks original file name and source line
number allowing simple parallel region identification. When the program finishes execution,
the elapsed time, the amount of energy used (in Joules), and the average computed
power (in Watts) of the parallel regions and the whole application are output. 
Additional information such as processor temperature is also available 
during application shut-down.
RCRtool runs on Intel architecture with the RAPL interface, and has been tested
on Sandy Bridge and Ivy Bridge implementations.

The overhead of the RCRdaemon is negligible on both architectures.  
It enables us to measure the energy consumption of the application 
with a granularity of about one millisecond. 

\subsection{RCRtool on Xeon Phi System}
The Intel MIC architecture in the Intel Xeon Phi chips is a recent addition to the
Intel processor offerings.
Our Phi accelerator cards contain 61 cores, each core supports 4 hardware threads. 
One notable feature is the 512-bit wide SIMD vectors providing fine-grain
vectorization and high floating-point performance for each thread.
With the wide vector registers, a single 
instruction can operate on 8 adjacent double-precision floating point data or 16 
single-precision floating point data. The cores, threads and vector unit combine
to achieve well over a Teraflop from a single socket.

RCRtool collects power information of Intel Phi natively
or on the host. Natively, users can track power usage in microWatts through    
a file (/sys/class/micras/power) updated every 50 millisecond.
RCRtool monitors the power at user level and computes the energy
consumption over time.
The information is available to the applications through the same simple API as on
Sandy Bridge.

RCRtool can also run on the host. On the host, it collects
power information using the MICAccessSDK API provided by Intel at the same 
granularity as the native version. 
Measurements in this paper were collected with a native Phi RCRdaemon.
 
\subsection{Polyhedral Optimizations Tools}
The Polyhedral Compiler Collection (PoCC)\cite{PoCC} was used to generate program 
variants with different optimizations.
The PoCC requires that programs contain static control parts (SCoP)\cite{SCoP1,SCoP2} so that 
valid transformations can be applied. Polybench is a collection of programs
that contain SCoPs and can be polyhedral optimized.  

PolyOpt (a Polyhedral Optimizer for the ROSE compiler)\cite{Polyopt} was also used to 
automatically detect SCoPs in applications. PolyOpt is integrated into the ROSE compiler.
Aside from its capability of extracting SCoP regions in an automatic way, it fully
supports PoCC analysis and optimizations. PolyOpt supports 
loops fusion, loop tiling, thread-level parallelization and vectorization. PolyOpt
has better support for side-effect free program features like math 
functions\cite{Math-Poly}, allowing some function calls within a SCoP.

PoCC generates hundreds and even thousands of program variants for simple programs, like Polybench.
PolyOpt, although more powerful than PoCC, still may not be able to extract any SCoPs
because of structural impediments. Changes to the program
may be required to  ``manually'' expose the SCoPs for PolyOpt, allowing 
loop transformations, parallelization, and vectorization to occur.
