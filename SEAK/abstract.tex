
\begin{abstract}
As the HPC community moves into the exascale computing era, application energy 
has become a big concern. Tuning for energy will be 
essential in the effort to overcome the limited power envelope. How is 
tuning for lower energy related to tuning for faster execution? 
Understanding that relationship can guide both performance and energy tuning
for exascale. In this paper, a strong correlation is presented between the
two that allows tuning for execution to be used as a proxy for 
energy tuning.  We also show that polyhedral compilers can effectively tune a realistic application 
for both time and energy.

For a large number of variants of the Polybench programs and LULESH energy consumption is strongly correlated with total execution time.
Optimizations can increase the power and energy required between variants, but variant with minimum execution time
also has the lowest energy usage.
The polyhedral framework was also used to optimize a 2D cardiac wave propagation simulation application.
Various loop optimizations including fusion, tiling, 
vectorization, and auto-parallelization, achieved a $20\%$ speedup over the 
baseline OpenMP implementation, with an equivalent reduction in energy 
on an Intel Sandy Bridge system. On an Intel Xeon Phi system, improvements
as high as 21\% in execution time and 19\% reduction in energy are obtained.

\end{abstract}
