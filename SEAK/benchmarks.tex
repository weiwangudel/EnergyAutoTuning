In this work, we evaluate three kinds of programs for energy auto-tuning with
polyhedral framework: Polybench programs, publicly-accessible LULESH 
program\cite{LULESH:versions}, and a realistic application developed and frequently 
used by our collaborators.

\subsection{Polybench}
Previous work has obtained significant speedups with the 
polyhedral framework for the Polybench programs\cite{EJ2011,EJ2012,EJ2013}.
Using PoCC, program variants were generated using
a different set of the 
optimizations from the following five groups: 
\begin{itemize}
    \item Loop fusion: smartfuse, maxfuse, nofuse
    \item Loop unrolling factor: 1, 2, 4, 8
    \item Loop tiling: 1, 16, 32, 64. Note that the number of different flags
 depends on the level of nested loops   
    \item Loop vectorization: on, off
    \item Loop parallelization: on, off
\end{itemize}
%\WW{We rely on Tiling Hyperplane method\cite{Hyperplane} to legally perform loop
%transformations. Loop fusion is performed to maximize locality by fusing statements. 
%As in\cite{EJ2013}, 1) nofuse means we do not fuse statements at all;
% 2)smartfuse means we only fuse together statements that carry data reuse and of 
%similar loop nesting depth; 3) maxfuse means we maximally fuse statements.
% }
The Tiling Hyperplane method\cite{Hyperplane} is used to legally perform loop transformations.
Loop fusion is performed to minimize loop overheads. Depending on reuse patterns, fusion can increase or decrease locality. As in\cite{EJ2013}, 1) nofuse results in no loop fusion 2) smartfuse only
fuses statements that carry data reuse and are at similar nesting levels 3) maxfuse performs all legal loop fusion.
If the maximum nested loop level is 3, applying all possible combinations
of the above flags generates 5135 program variants. The ROSE source-to-source
compiler was used to add energy profiling calls to each variant.
GCC(4.4.6) generated the final executable. During execution,
 periodic queries to the RCRtool blackboard provide the energy consumption information.
Figure~\ref{fig:Workflow} gives the workflow for measuring energy consumption of 
Polybench programs using the energy-aware polyhedral compiler framework.

