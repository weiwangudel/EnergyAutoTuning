Reducing application energy consumption is important to improve user experience of 
embedded systems, e.g. smart phones, GPS navigators etc. 
Tuning applications for better energy efficiency is faced with challenges brought by
diverse architectures, difficulty of obtaining fine-grain measurements of power, 
as well as enormous amount of tuning choices.
Previous work which performed coarse-grain measurement have provided evidence for the 
existence of opportunities to auto-tune for energy in parallel applications\cite{Tiwari:EnergyAutoTune}.
This work uses the RCRtool \cite{us}, a fine-grain energy measurement tool, to easily attribute energy consumption to particular
regions of application kernels and to serve energy tuning. 
When tuning the application for
better execution time and energy usage, some combination of loop optimizations, including loop 
tiling, loop unrolling, and loop fusion, are usually performed on kernels along with the auto-parallelization.
Determining which set of optimizations produces the best results is hard.
Polyhedral auto-tuning frameworks have shown promising results at simplifying that effort\cite{EJ2012}
for small computation kernels like the Polybench programs. 
This work investigates the effectiveness of such framework to tune for 
energy on two different multi-core architectures: Intel Sandy Bridge
and Many Integrated Core (MIC).
 
This paper has two main contributions: 
1)Performance speedups and energy savings on Sandy Bridge and MIC for 
various Polybench kernels applying loop transformations. 
2)Evaluation of architectural differences of energy behaviours corresponding to the same loop transformations.

The rest of the paper is organized as follows. In Section~\ref{sec:tools}, we describe the tools
used to measure energy consumptions. Section~\ref{sec:benchmarks} describes the benchmarks used for  
measuring the energy consumptions. Experimental setup, results and analysis are presented in Section~\ref{sec:setup} and Section~\ref{sec:results}. Section~\ref{sec:conclusion} has our conclusions.

