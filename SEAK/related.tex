%\todo{Needs more work to make sense}
\subsection{Energy Measurement and Tuning}
The accuracy of our energy-aware polyhedral framework relies on the exactness
of RAPL. While presenting the HAECER framework for short-term energy measurements 
using RAPL, H\"{a}hnel {\it et al.} reported the identical
curve characteristics comparing RAPL with external measurement\cite{RAPL-Related}.
The HAECER framework was not used here because of the need to measure long code
paths--the executions finish in the order of seconds by testing with large datasets.


Rahman {\it et al.}\cite{CF12} studied the impact of application level optimizations from both the 
performance and power efficiency perspective of various applications. They found
that optimizing for performance did not guarantee better power consumption. We 
observed similar results in Figure~\ref{fig:TE} and Figure~\ref{fig:2mm-TE} for non-optimal
program variants but the graphs showed that for the optimal case, tuning for performance and power were 
effectively equivalent. To improve performance and energy efficiency for a Many-Core architecture,
Garcia {\it et al.}\cite{Garcia} studied the energy consumptions of applications and proposed models 
characterizing application energy consumption footprints. We did not
develop energy models but took advantage of the exposed hardware interfaces to obtain accurate   
energy consumption information from modern commodity processor architectures like
Intel Sandy Bridge and Xeon Phi.

To improve performance, people have 
developed techniques from distinctive ways. Tavarageri {\it et al.}\cite{Reduce-Cache} 
adopted compiler analysis approach to configure the cache size to reduce energy consumption
without performance loss. New programming languages\cite{IPDPS13:LULESH} and 
models like Chapel, Liszt and others were introduced to facilitate program optimizations
on parallel architectures.
