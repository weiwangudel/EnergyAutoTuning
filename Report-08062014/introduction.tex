Dynamic Voltage and Frequency Scaling (DVFS) and Duty Cycle Modulation (DCM)
are two power saving techinques on Intel Sandy Bridge processors. 
DVFS saves power by reducing the frequency and voltage supply of the chip. 
Duty Cycle Modulation reduces power consumption by freezing 
the cores for a specified time interval -- the effect is that the machine frequency
is reduced (without changing the voltage).   
DVFS benefits (in terms of energy) coarse-grained loops more than fine-grain loops,
because it has larger transition overhead than DCM. 
Duty Cycle Modulation has fast transition but it only affects frequency -- 
the power savings might not be as significant as DVFS reduced to the same frequency.
Because of the faster transition, DCM has more potential in controlling fine-grained code regions
than DVFS.

In order to optimize applications for energy, we experimented 
with the community detection application in presence of frequency
scaling and duty cycle modulation. We found that running the program at a 
lower frequency yielded the minimum EDP (energy-delay product) 
without being the minimum energy, as shown in Figure~1.
We thought it might be due to memory behaviour of the application.

Then we looked at memory counters and found that this counter can in part determine a
 frequency (or duty cycle modulation level) that will lead to the best energy.
We also tried to characterize loops and correlate the characteristics to the 
Frequency/ClockModulation setting. We hope that by running different loops at 
different low power mode we can achieve a minimum Energy (E) or Energy Delay Product (EDP).


