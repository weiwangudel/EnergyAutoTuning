As expected, using an auto-tuning framework on a variety of
small benchmarks and small applications has shown the high
degree of correlation between execution time and energy
consumption. Individual optimization can however
have significant impact on the power required by an
application. In the Polybench program, \emph{covariance},
using the ``maxfuse'' option resulted in a 20+\% percent power
increase. With the correct tile size ``maxfuse'' also
resulted in the 50+\% time decrease. ``maxfuse'' increases power 
consumption but reduces total energy required to complete the computation due to
the decrease in execution time.
Understanding how power and energy are used at the small
scale can contribute to the understanding of power/energy requirements of Exascale applications.

Polyhedral optimization techniques can provide significant 
increases in performance but currently require 
significant user modifications to any real application
to generate SCoPs with reasonable compilation times.
On small real applications, like LULESH and \emph{brdr2d},
polyhedral transformations allow the discovery of
effective tiling sizes for SCoPs within the applications.

