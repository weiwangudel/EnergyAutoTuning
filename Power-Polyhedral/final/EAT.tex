% This is "sig-alternate.tex" V2.0 May 2012
% This file should be compiled with V2.5 of "sig-alternate.cls" May 2012
%
% This example file demonstrates the use of the 'sig-alternate.cls'
% V2.5 LaTeX2e document class file. It is for those submitting
% articles to ACM Conference Proceedings WHO DO NOT WISH TO
% STRICTLY ADHERE TO THE SIGS (PUBS-BOARD-ENDORSED) STYLE.
% The 'sig-alternate.cls' file will produce a similar-looking,
% albeit, 'tighter' paper resulting in, invariably, fewer pages.
%
% ----------------------------------------------------------------------------------------------------------------
% This .tex file (and associated .cls V2.5) produces:
%       1) The Permission Statement
%       2) The Conference (location) Info information
%       3) The Copyright Line with ACM data
%       4) NO page numbers
%
% as against the acm_proc_article-sp.cls file which
% DOES NOT produce 1) thru' 3) above.
%
% Using 'sig-alternate.cls' you have control, however, from within
% the source .tex file, over both the CopyrightYear
% (defaulted to 200X) and the ACM Copyright Data
% (defaulted to X-XXXXX-XX-X/XX/XX).
% e.g.
% \CopyrightYear{2007} will cause 2007 to appear in the copyright line.
% \crdata{0-12345-67-8/90/12} will cause 0-12345-67-8/90/12 to appear in the copyright line.
%
% ---------------------------------------------------------------------------------------------------------------
% This .tex source is an example which *does* use
% the .bib file (from which the .bbl file % is produced).
% REMEMBER HOWEVER: After having produced the .bbl file,
% and prior to final submission, you *NEED* to 'insert'
% your .bbl file into your source .tex file so as to provide
% ONE 'self-contained' source file.
%
% ================= IF YOU HAVE QUESTIONS =======================
% Questions regarding the SIGS styles, SIGS policies and
% procedures, Conferences etc. should be sent to
% Adrienne Griscti (griscti@acm.org)
%
% Technical questions _only_ to
% Gerald Murray (murray@hq.acm.org)
% ===============================================================
%
% For tracking purposes - this is V2.0 - May 2012
\documentclass{sig-alternate}
%\usepackage{algorithm}
%\usepackage{algorithmic}
\usepackage{color}
\usepackage{url}
\usepackage{hyperref} 

\newcommand{\todo}[1]{%                                                          
  \noindent {\bf {\color {red} TODO: #1 }}%                                      
} 
\newcommand{\WW}[1]{%                                                          
  \noindent {\bf {\color {blue} #1}}%                                      
} 

\usepackage{subfigure}
\pagenumbering{arabic}
\toappear{
\hrule \vspace{5pt}
IMPACT 2014\\
Fourth International Workshop on Polyhedral Compilation Techniques\\
Jan 20, 2014, Vienna, Austria\\
In conjunction with HiPEAC 2014.\\[10pt]
\url{http://impact.gforge.inria.fr/impact2014}\\
}

\begin{document}
%
% --- Author Metadata here ---
%\conferenceinfo{WOODSTOCK}{'97 El Paso, Texas USA}
%\CopyrightYear{2007} % Allows default copyright year (20XX) to be over-ridden - IF NEED BE.
%\crdata{0-12345-67-8/90/01}  % Allows default copyright data (0-89791-88-6/97/05) to be over-ridden - IF NEED BE.
% --- End of Author Metadata ---

\title{Energy Auto-tuning using Polyhedral Approach}

%
% You need the command \numberofauthors to handle the 'placement
% and alignment' of the authors beneath the title.
%
% For aesthetic reasons, we recommend 'three authors at a time'
% i.e. three 'name/affiliation blocks' be placed beneath the title.
%
% NOTE: You are NOT restricted in how many 'rows' of
% "name/affiliations" may appear. We just ask that you restrict
% the number of 'columns' to three.
%
% Because of the available 'opening page real-estate'
% we ask you to refrain from putting more than six authors
% (two rows with three columns) beneath the article title.
% More than six makes the first-page appear very cluttered indeed.
%
% Use the \alignauthor commands to handle the names
% and affiliations for an 'aesthetic maximum' of six authors.
% Add names, affiliations, addresses for
% the seventh etc. author(s) as the argument for the
% \additionalauthors command.
% These 'additional authors' will be output/set for you
% without further effort on your part as the last section in
% the body of your article BEFORE References or any Appendices.

\numberofauthors{2} %  in this sample file, there are a *total*
% of EIGHT authors. SIX appear on the 'first-page' (for formatting
% reasons) and the remaining two appear in the \additionalauthors section.
%
\author{
% You can go ahead and credit any number of authors here,
% e.g. one 'row of three' or two rows (consisting of one row of three
% and a second row of one, two or three).
%
% The command \alignauthor (no curly braces needed) should
% precede each author name, affiliation/snail-mail address and
% e-mail address. Additionally, tag each line of
% affiliation/address with \affaddr, and tag the
% e-mail address with \email.
%
\alignauthor Wei Wang, John Cavazos  \\
       \affaddr{University of Delaware}\\                          
       \affaddr{101 Smith Hall} \\
       \affaddr{Newark, DE 19702}\\                                         
       \email{{weiwang,cavazos}@udel.edu}                                      
\alignauthor Allan Porterfield         \\                                                 
       \affaddr{University of North Carolina - RENCI}\\                          
       \affaddr{100 Europa Dr}\\                                                 
       \affaddr{Chapel Hill, NC 27517}\\                                         
       \email{akp@renci.org}                                      
%\alignauthor
% 1st. author
%\alignauthor
%Ben Trovato\titlenote{Dr.~Trovato insisted his name be first.}\\
%       \affaddr{Institute for Clarity in Documentation}\\
%       \affaddr{1932 Wallamaloo Lane}\\
%       \affaddr{Wallamaloo, New Zealand}\\
%       \email{trovato@corporation.com}
%% 2nd. author
%\alignauthor
%G.K.M. Tobin\titlenote{The secretary disavows
%any knowledge of this author's actions.}\\
%       \affaddr{Institute for Clarity in Documentation}\\
%       \affaddr{P.O. Box 1212}\\
%       \affaddr{Dublin, Ohio 43017-6221}\\
%       \email{webmaster@marysville-ohio.com}
%% 3rd. author
%\alignauthor Lars Th{\o}rv{\"a}ld\titlenote{This author is the
%one who did all the really hard work.}\\
%       \affaddr{The Th{\o}rv{\"a}ld Group}\\
%       \affaddr{1 Th{\o}rv{\"a}ld Circle}\\
%       \affaddr{Hekla, Iceland}\\
%       \email{larst@affiliation.org}
%\and  % use '\and' if you need 'another row' of author names
%% 4th. author
%\alignauthor Lawrence P. Leipuner\\
%       \affaddr{Brookhaven Laboratories}\\
%       \affaddr{Brookhaven National Lab}\\
%       \affaddr{P.O. Box 5000}\\
%       \email{lleipuner@researchlabs.org}
%% 5th. author
%\alignauthor Sean Fogarty\\
%       \affaddr{NASA Ames Research Center}\\
%       \affaddr{Moffett Field}\\
%       \affaddr{California 94035}\\
%       \email{fogartys@amesres.org}
%% 6th. author
%\alignauthor Charles Palmer\\
%       \affaddr{Palmer Research Laboratories}\\
%       \affaddr{8600 Datapoint Drive}\\
%       \affaddr{San Antonio, Texas 78229}\\
%       \email{cpalmer@prl.com}
%}
%% There's nothing stopping you putting the seventh, eighth, etc.
%% author on the opening page (as the 'third row') but we ask,
%% for aesthetic reasons that you place these 'additional authors'
%% in the \additional authors block, viz.
%\additionalauthors{Additional authors: John Smith (The Th{\o}rv{\"a}ld Group,
%email: {\texttt{jsmith@affiliation.org}}) and Julius P.~Kumquat
%(The Kumquat Consortium, email: {\texttt{jpkumquat@consortium.net}}).}
}
%\date{30 July 1999}
% Just remember to make sure that the TOTAL number of authors
% is the number that will appear on the first page PLUS the
% number that will appear in the \additionalauthors section.

\maketitle


\begin{abstract}
As the HPC community moves into the exascale computing era, application energy 
has become a big concern. Tuning for energy will be 
essential for the embedded systems to meet. in the effort to overcome the limited power envelope. How is 
tuning for lower energy related to tuning for faster execution? 
Understanding that relationship can guide both performance and energy tuning
for exascale. In this paper, a strong correlation is presented between the
two that allows tuning for execution to be used as a proxy for 
energy tuning.  We also show that polyhedral compilers can effectively tune a realistic application 
for both time and energy. \\
For a large number of variants of the Polybench programs
Optimizations can increase the power and energy required between variants, but variant with minimum execution time
also has the lowest energy usage.
Various loop optimizations including fusion, tiling, 
vectorization, and auto-parallelization, achieved a $20\%$ speedup over the 
baseline OpenMP implementation, with an equivalent reduction in energy 
on an Intel Sandy Bridge system. On an Intel Xeon Phi system, improvements
as high as 21\% in execution time and 19\% reduction in energy are obtained.

\end{abstract}


\section{Introduction}
\label{sec:intro}
Dynamic Voltage and Frequency Scaling (DVFS) and Duty Cycle Modulation (DCM)
are two power saving techinques on Intel Sandy Bridge processors. 
DVFS saves power by reducing the frequency and voltage supply of the chip. 
Duty Cycle Modulation reduces power consumption by freezing 
the cores for a specified time interval -- the effect is that the machine frequency
is reduced (without changing the voltage).   
DVFS benefits (in terms of energy) coarse-grained loops more than fine-grain loops,
because it has larger transition overhead than DCM. 
Duty Cycle Modulation has fast transition but it only affects frequency -- 
the power savings might not be as significant as DVFS reduced to the same frequency.
Because of the faster transition, DCM has more potential in controlling fine-grained code regions
than DVFS.

In order to optimize applications for energy, we experimented 
with the community detection application in presence of frequency
scaling and duty cycle modulation. We found that running the program at a 
lower frequency yielded the minimum EDP (energy-delay product) 
without being the minimum energy, as shown in Figure~1.
We thought it might be due to memory behaviour of the application.

Then we looked at memory counters and found that this counter can in part determine a
 frequency (or duty cycle modulation level) that will lead to the best energy.
We also tried to characterize loops and correlate the characteristics to the 
Frequency/ClockModulation setting. We hope that by running different loops at 
different low power mode we can achieve a minimum Energy (E) or Energy Delay Product (EDP).




\section{Energy Measurement-and-Tuning Tools}
\label{sec:tools}
To understand energy consumption, execution time and various optimizations,
a light-weight fine-grained measurement RCRtool is required. 
Finding the optimal combination of compiler optimizations 
requires a compilation framework, like the Polyhedral Compiler
Collection, that easily produces a large number of 
program variants with specific optimization parameters.
\subsection{RCRtool}
The Intel Sandy Bridge architecture allows users to track energy usage through 
the exposed Running Average Power Limit (RAPL) hardware counter. 
The RCRtool records the current value of the counter at least 1000 times a second by writing 
into a shared-memory data structure. This ``blackboard'' structure provides a hierarchical view of the system
where various current performance information is stored. The 
information is available to any OpenMP applications through a simple API that 
delineates a code region for measurement with a start and end call.
When the program finishes execution,
the elapsed time, the amount of energy used (in Joules), and the average computed
power (in Watts) of the kernel regions and the whole application are output. 
\subsection{RCRtool on Xeon Phi System}
RCRtool collects power information of Intel Phi natively. 
Users can track power usage in microWatts through    
a file (/sys/class/micras/power) updated every 50 millisecond.
RCRtool monitors the power at user level and computes the energy
consumption over time.
The information is available to the applications through the same simple API as on
Sandy Bridge.
\subsection{Polyhedral Optimizations Tools}
The Polyhedral Compiler Collection (PoCC) was used to generate program 
variants with different optimizations.
The PoCC requires that programs contain static control parts (SCoP) so that 
valid transformations can be applied. Polybench is a collection of programs
that contain SCoPs and can be polyhedral optimized.  


\section{Benchmarks for Energy Measurement and Tuning}
\label{sec:benchmarks}
In this work, we evaluate three kinds of programs for energy auto-tuning with
polyhedral framework: Polybench programs, publicly-accessible LULESH 
program\cite{LULESH:versions}, and a realistic application developed and frequently 
used by our collaborators.

\subsection{Polybench}
Previous work has obtained significant speedups with the 
polyhedral framework for the Polybench programs\cite{EJ2011,EJ2012,EJ2013}.
Extending that work to examine whether the best tuned variants are also
the most energy efficient is the focus of this work.
Using PoCC, program variants were generated using
a different set of the 
optimizations from the following five groups: 
\begin{itemize}
    \item Loop fusion: smartfuse, maxfuse, nofuse
    \item Loop unrolling factor: 1, 2, 4, 8
    \item Loop tiling: 1, 16, 32, 64. Note that the number of different flags
 depends on the level of nested loops   
    \item Loop vectorization: on, off
    \item Loop parallelization: on, off
\end{itemize}
If the maximum nested loop level is 3, applying all possible combinations
of the above flags generates 5135 program variants. The ROSE source-to-source
compiler was used to add energy profiling calls to each variant.
GCC(4.4.6) generated the final executable. During execution,
 periodic queries to the RCRtool blackboard provide the energy consumption information.
Figure~\ref{fig:Workflow} gives the workflow for measuring energy consumption of 
Polybench programs using the energy-aware polyhedral compiler framework.

\begin{figure}[t]
    %\rotatebox{-90}{\includegraphics[height=3in]{TE}}
    \includegraphics[width=3.4in]{workflow}
    \caption{Graph showing the workflow of obtaining energy consumption of polyhedral
optimized (Polybench) programs. }
    \label{fig:Workflow}
\end{figure}

\subsection{LULESH}
LULESH\cite{LULESH:versions} is a shock hydrodynamic simulation application. It
mimics a larger realistic application called ALE3D. We used the OpenMP implementation
v1.0 for evaluation. The original LULESH uses a 
block structured mesh accessed via an indirect reference pattern\cite{LULESH:versions}.
In order to make LULESH go through the polyhedral compilation procedure, we modified
LULESH by resolving all indirect array accesses. Although doing this oversimplified
LULESH, it allows us to study the energy and time relationship of polyhedral 
compilation techniques with LULESH. 

LULESH OpenMP implementation contains 30 parallel regions, 6 of which take up more
than $60\%$ of the total application time\cite{us}. We manually converted two most 
significant parallel regions to two SCoPs so that they can be passed to polyhedral
framework. The resulting largest SCoP contained too many dependences and we found
it was hard for the polyhedral compiler to finish transformation and parallelization.
When all temporary variables were eliminated from the most computationally intensive loop
to create an SCoP, greatly expanded statements required hours of compilation to finish generating even one variant.
%Even when all the dependences are eliminated inside the largest SCoP, it still took the polyhedral
%compiler hours to finish one transformation.
In this work, we focus on optimizing 
the 2nd (largest) SCoP of LULESH. 200 program variants were produced 
by applying loop fusion (maxfuse and 
smartfuse), loop tiling, vectorization, and parallelization.
The execution time and energy of each was measured.

\subsection{A Realistic Application}
In addition to the Polybench programs, a 2D monodomain cardiac wave propagation simulation 
(named \emph{brdr2d}) was used as a test case. Its model involves
solving a set of ODEs and PDEs and is well-known in the computational 
cardiac modeling field\cite{Me}. Equation (1) is the PDE that needs to be 
solved. The ODEs are used to represent the $I_{ion}$ variable.
\begin{equation}
C_{m}\frac{\partial V_{m}}{\partial t} = \nabla \cdot D\nabla{V_{m}}-I_{ion}
\end{equation}
The sequential C implementation is more than 1K lines. 

One loop nest takes up more than 90\% of the total application execution time. 
The dominate loop nest is an ideal situation for the polyhedral compiler. 
The loop nest is inside a while loop and is executed many times. 
This code structure is not unique to cardiac wave propagation simulation. 
Computationally dominate loops inside either while loops looking for 
some termination condition or inside a simulation time-step loop
are common in scientific codes. LULESH falls into this category
with multiple loop nests within a time-step loop.

While PolyOpt originally cannot extract any SCoP from \emph{brdr2d}, it does output
information useful to the user to manually transform the application to contain at least one SCoP.
To expose the SCoPs the following changes were required. The computation part 
of \emph{brdr2d} was fully inlined removing all function calls.
Then, all array indexes were changed to be affine functions of the 
loop iterators. This involved loop unswitching to specialize modular operations
like $step~\%~2$. Finally, the number of dependencies was reduced by forward substitution
of temporary variables. After these changes, PolyOpt automatically detected the code
region and applied various transformations to the SCoPs. 

%Different from LULESH, the forward substitution did not result in 
%extreme expansion of code when converting the nested loops into SCoPs. Thus, the 
%polyhedral compiler generated program variants in a timely manner.

Different program variants were generated to explore data locality and parallelism
using loop fusion (smartfuse/maxfuse), different tiling sizes, vectorization and auto-parallelization. 
OpenMP pragmas were automatically generated for each variant.
The original sequential C implementation had OpenMP pragmas manually added to serve as
a baseline.
Four different input files for \emph{brdr2d} were used to study how the
performance of the program variants is impacted by different
input sizes. 
 


\section{Experimental Setup}
\label{sec:setup}
The tests ran on a 2-socket 8-core Intel Xeon E5-2680 processor with 20MB (40MB total) L3 cache.
PoCC v1.2 was used to generate program variants from Polybench v3.2.
The extra large data set (specified in Polybench) was used. 
\todo{ICC version?} was the backend compiler. Every executable was compiled with
-O3 optimization flag.  

Experiments were also run on a Xeon Phi coprocessor. 
The Phi architecture accelerator card contained 61 cores clocked at 1.09GHz. Each core had 512KB of L2 cache. 
The generated program variants used ICC v14.0.0 compiler as their backend, producing OpenMP programs that ran natively on the Phi.


\section{Experimental Results}
\label{sec:results}
First we report the speedups and energy saving achieved on two architectures applying
loop optimizations. Then we show loop optimizations that work the best on
one architecture do not necessarily work the best on the other, in terms of
time taken and energy consumed.
\subsection{Speedups and Energy Savings}

\subsection{Cross Architecture Comparsion}

\begin{figure}
\centering
\subfigure[Best SandyBridge Optimization Sequences on Xeon Phi] {                        
  %\includegraphics[width=0.45\textwidth]{BestMIConSNB}   
  \includegraphics{BestSNBonMIC}
  \label{Sandy}
} 
\subfigure[Best MIC Optimization Sequences on SandyBridge] {               
  %\includegraphics[width=0.45\textwidth]{BestSNBonMIC}
  \includegraphics{BestMIConSNB}   
  \label{MIC}
} 
\caption{
}                   
\label{fig:Brdr2d-TE}                                                   
\end{figure} 

Figure~\ref{Sandy} shows the best optimization sequences of the six benchmarks
chosen from Sandy Bridge do not perform as good when run on the MIC architecture. 
Comparing with the optimal optimization sequence on the MIC, the Sandy Bridge
sequences incurred significant increase in execution time and energy consumption
for 2mm, covariance, and gemm. The Sandy Bridge optimization sequences 
performed as good as the optimal sequence on the MIC for the other three benchmarks. 
Figure~\ref{MIC} compares the time and energy (running on Sandy Bridge) of 
the best optimization sequences chosen from MIC with those of the optimal optimization 
sequences chosen from Sandy Bridge. 2mm, covariance, gemm, and seidel-2d consumed
from ~200\% to ~350\% more energy and time. 


\section{Related Work}
\label{sec:related}
Reducing Energy Usage with Memory and Computation-Aware Dynamic Frequency Scaling
characterizes applications using static analysis and runtime tracing that automatically
acquires application signatures - characterizations of the patterns of execution of 
each loop in an application. The characterization is matched with a set of benchmark 
loops, which have been fully explored for optimal frequency setting. 

It would be good if we can test Alexandra's Decoupled Access-Execute model, 
where they called for faster DVFS transition and \textit{modelled} faster DVFS 
for fine-grained access phase. 
We should test whether Duty Cycle Modulation is a good alternative to waiting for 
DVFS transition to be close to instantaneous. 



\section {Conclusion}
\label{sec:conclusion}
In this paper, we tuned six representative Polybench kernels for energy on
an Intel SandyBridge processor and an Intel Xeon Phi coprocessor by applying various loop 
transformations. For the \texttt{2mm} and \texttt{gemm} kernels (dense matrix kernels),
we observed \emph{non-correlated} speedups and energy savings over the baseline version,
i.e. \emph{fastest execution does not guarantee fewest energy consumption}. 
We also showed that good loop transformations for one architecture do not carry over to
other architectures.


%\balancecolumns

%ACKNOWLEDGMENTS are optional
\section{Acknowledgments}
This work is supported by the DOE XPress (DE-SC0008704) and the DoD ATPAR (PNNL-214990).
The authors would like to thank EunJung Park and Matthew Kay for sharing the polyhedral optimized 
Polybench programs and the cardiac wave propagation simulation application. The 
authors would also like to thank Louis-No{\"e}l Pouchet,
Albert Cohen, and Riyadh Baghdadi for their help with the polyhedral compilation 
of benchmarks. Last but not least, the authors would like to thank all the anonymous 
reviewers for their valuable feedback to help improve this work.

%\balancecolumns
%
% The following two commands are all you need in the
% initial runs of your .tex file to
% produce the bibliography for the citations in your paper.
%\bibliographystyle{abbrv}
%\bibliography{EAT}  % sigproc.bib is the name of the Bibliography in this case
% You must have a proper ".bib" file
%  and remember to run:
% latex bibtex latex latex
% to resolve all references
%
% ACM needs 'a single self-contained file'!
%

\begin{thebibliography}{10}                                                      
                                                                                 
\bibitem{Math-Poly}                                                              
M.-W. Benabderrahmane, L.-N. Pouchet, A.~Cohen, and C.~Bastoul.                  
\newblock The polyhedral model is more widely applicable than you think.         
\newblock In {\em Proceedings of the 19th joint European conference on Theory    
  and Practice of Software, international conference on Compiler Construction},  
  CC'10/ETAPS'10, pages 283--303, 2010.                                          
                                                                                 
\bibitem{Hyperplane}                                                             
U.~Bondhugula, A.~Hartono, J.~Ramanujam, and P.~Sadayappan.                      
\newblock {A Practical Automatic Polyhedral Parallelizer and Locality            
  Optimizer}.                                                                    
\newblock In {\em Proceedings of the 2008 ACM SIGPLAN Conference on Programming  
  Language Design and Implementation}, PLDI '08, pages 101--113, 2008.           
                                                                                 
\bibitem{Cho}                                                                    
S.~Cho and R.~Melhem.                                                            
\newblock {On the Interplay of Parallelization, Program Performance, and Energy  
  Consumption}.                                                                  
\newblock {\em Parallel and Distributed Systems, IEEE Transactions on},          
  21(3):342--353, 2010.                                                          
                                                                                 
\bibitem{SCoP1}                                                                  
P.~Feautrier.                                                                    
\newblock {Some efficient solutions to the affine scheduling problem. Part II.   
  Multidimensional time}.                                                        
\newblock {\em International Journal of Parallel Programming}, 21(6):389--420,   
  1992.                                                                          
                                                                                 
\bibitem{Garcia}                                                                 
E.~Garcia and G.~Gao.                                                            
\newblock Strategies for improving performance and energy efficiency on a        
  many-core.                                                                     
\newblock In {\em Proceedings of the ACM International Conference on Computing   
  Frontiers}, CF '13, pages 9:1--9:4, 2013.     

\bibitem{SCoP2}                                                                  
S.~Girbal, N.~Vasilache, C.~Bastoul, A.~Cohen, D.~Parello, M.~Sigler, and        
  O.~Temam.                                                                      
\newblock Semi-automatic composition of loop transformations for deep            
  parallelism and memory hierarchies.                                            
\newblock {\em Int. J. Parallel Program.}, 34(3):261--317, June 2006.            
                                                                                 
\bibitem{Scott}                                                                  
S.~Grauer-Gray, L.~Xu, R.~Searles, S.~Ayalasomayajula, and J.~Cavazos.           
\newblock {Auto-tuning a high-level language targeted to GPU codes}.             
\newblock In {\em Innovative Parallel Computing (InPar), 2012}, pages 1--10,     
  2012.                                                                          
                                                                                 
\bibitem{RAPL-Related}                                                           
M.~H\"{a}hnel, B.~D\"{o}bel, M.~V\"{o}lp, and H.~H\"{a}rtig.                     
\newblock {Measuring Energy Consumption for Short Code Paths Using RAPL}.        
\newblock {\em SIGMETRICS Perform. Eval. Rev.}, 40(3):13--17, 2012.              
                                                                                 
\bibitem{IntelSystemProgrammingVol3}                                             
{Intel}.                                                                         
\newblock {Intel 64 and {IA-32} Architectures Software Developer's Manual,       
  Volume 3}.                                                                     
\newblock \url{http://download.intel.com/products/processor/manual/253669.pdf},  
  2013.                                                                          
                                                                                 
\bibitem{LULESH:versions}                                                        
I.~Karlin, A.~Bhatele, B.~L. Chamberlain, J.~Cohen, Z.~Devito, M.~Gokhale,       
  R.~Haque, R.~Hornung, J.~Keasler, D.~Laney, E.~Luke, S.~Lloyd, J.~McGraw,      
  R.~Neely, D.~Richards, M.~Schulz, C.~H. Still, F.~Wang, and D.~Wong.           
\newblock {LULESH Programming Model and Performance Ports Overview}.             
\newblock Technical Report LLNL-TR-608824, December 2012.    

\bibitem{IPDPS13:LULESH}                                                         
I.~Karlin, A.~Bhatele, J.~Keasler, B.~L. Chamberlain, J.~Cohen, Z.~DeVito,       
  R.~Haque, D.~Laney, E.~Luke, F.~Wang, D.~Richards, M.~Schulz, and C.~Still.    
\newblock {Exploring Traditional and Emerging Parallel Programming Models using  
  a Proxy Application}.                                                          
\newblock In {\em 27th IEEE International Parallel \& Distributed Processing     
  Symposium (IEEE IPDPS 2013)}, Boston, USA, May 2013.                           
                                                                                 
\bibitem{LCPC2013}                                                               
A.~Konstantinidis, P.~H. Kelly, J.~Ramanujam, and P.~Sadayappan.                 
\newblock {Parametric GPU Code Generation for Affine Loop Programs}.             
\newblock In {\em Proceedings of the 26th International Conference on Languages  
  and Compilers for Parallel Computing}, LCPC'13, 2013.                          
                                                                                 
\bibitem{EJ2012}                                                                 
E.~Park, J.~Cavazos, and M.~A. Alvarez.                                          
\newblock Using graph-based program characterization for predictive modeling.    
\newblock In {\em CGO}, pages 196--206, 2012.                                    
                                                                                 
\bibitem{EJ2013}                                                                 
E.~Park, J.~Cavazos, L.-N. Pouchet, C.~Bastoul, A.~Cohen, and P.~Sadayappan.     
\newblock {Predictive Modeling in a Polyhedral Optimization Space}.              
\newblock {\em International Journal of Parallel Programming}, 41(5):704--750,   
  2013.               

\bibitem{EJ2011}                                                                 
E.~Park, L.-N. Pouche, J.~Cavazos, A.~Cohen, and P.~Sadayappan.                  
\newblock Predictive modeling in a polyhedral optimization space.                
\newblock In {\em Proceedings of the 9th Annual IEEE/ACM International           
  Symposium on Code Generation and Optimization}, CGO '11, pages 119--129,       
  Washington, DC, USA, 2011. IEEE Computer Society.                              
                                                                                 
\bibitem{us}                                                                     
A.~Porterfield, R.~Fowler, S.~Bhalachandra, and W.~Wang.                         
\newblock {OpenMP and MPI Application Energy Measurement Variation}.             
\newblock In {\em Proceedings of the 1st International Workshop on Energy        
  Efficient Supercomputing}, E2SC '13, pages 7:1--7:8, 2013.                     
                                                                                 
\bibitem{PoCC}                                                                   
L.-N. Pouchet.                                                                   
\newblock {PoCC: the Polyhedral Compiler Collection}.                            
\newblock \url{http://www.cs.ucla.edu/~pouchet/software/pocc/}, 2013.            
                                                                                 
\bibitem{Polybench}                                                              
L.-N. Pouchet.                                                                   
\newblock {Polybench}.                                                           
\newblock \url{http://www.cse.ohio-state.edu/~pouchet/software/polybench/},      
  2013.                                                                          
\balancecolumns
                                                                                 
\bibitem{Polyopt}                                                                
L.-N. Pouchet.                                                                   
\newblock {PolyOpt/C: a Polyhedral Optimizer for the ROSE compiler}.             
\newblock \url{http://www.cs.ucla.edu/~pouchet/software/polyopt/}, 2013.         
                                                                                 
\bibitem{ROSE}                                                                   
D.~J. Quinlan and C.~L. Liao.                                                    
\newblock {ROSE Compiler}.                                                       
\newblock \url{http://rosecompiler.org/}, 2013.    

\bibitem{CF12}                                                                   
S.~M.~F. Rahman, J.~Guo, A.~Bhat, C.~Garcia, M.~H. Sujon, Q.~Yi, C.~Liao, and    
  D.~Quinlan.                                                                    
\newblock Studying the impact of application-level optimizations on the power    
  consumption of multi-core architectures.                                       
\newblock In {\em Proceedings of the 9th conference on Computing Frontiers}, CF  
  '12, pages 123--132, 2012.                                                     
                                                                                 
\bibitem{Reduce-Cache}                                                           
S.~Tavarageri and P.~Sadayappan.                                                 
\newblock {A Compiler Analysis to Determine Useful Cache Size for Energy         
  Efficiency}.                                                                   
\newblock In {\em Proceedings of the 2013 IEEE 27th International Symposium on   
  Parallel and Distributed Processing Workshops and PhD Forum}, IPDPSW '13,      
  pages 923--930, 2013.                                                          
                                                                                 
\bibitem{Tiwari:EnergyAutoTune}                                                  
A.~Tiwari, M.~A. Laurenzano, L.~Carrington, and A.~Snavely.                      
\newblock Auto-tuning for energy usage in scientific applications.               
\newblock In {\em Proceedings of the 2011 international conference on Parallel   
  Processing - Volume 2}, Euro-Par'11, pages 178--187, Berlin, Heidelberg,       
  2012. Springer-Verlag.                                                         
                                                                                 
\bibitem{SWIM}                                                                   
N.~Vasilache, C.~Bastoul, and A.~Cohen.                                          
\newblock Polyhedral code generation in the real world.                          
\newblock In {\em In Proceedings of the International Conference on Compiler     
  Construction (ETAPS CC'06), LNCS}, pages 185--201. Springer-Verlag, 2006.      
                                                                                 
\bibitem{Me}                                                                     
W.~Wang, H.~Huang, M.~Kay, and J.~Cavazos.                                       
\newblock {GPGPU} accelerated cardiac arrhythmia simulations.                    
\newblock In {\em Engineering in Medicine and Biology Society,EMBC, 2011 Annual  
  International Conference of the IEEE}, pages 724--727, 2011.                   
                                                                
\bibitem{SpeedEnergy}                                                            
T.~Yuki and S.~Rajopadhye.                                                       
\newblock {Folklore Confirmed: Compiling for Speed = Compiling for Energy}.      
\newblock In {\em Proceedings of the 26th International Conference on Languages  
  and Compilers for Parallel Computing}, LCPC'13, 2013.                          
                                                                                 
\end{thebibliography}   


% That's all folks!
\end{document}
