As the HPC community approaches the exascale computing era, reducing application energy 
and power consumption is important. The cost of supplying the peak power and 
operational energy for exascale systems will be substantial. 
Controlling the application energy use of the increasingly powerful
compute nodes will be required. 
Before tuning an application for power efficiency, it is necessary to 
understand how energy consumption is related to the application execution time.  
Knowledge of the relationship between performance and energy can guide the tuning effort.

Previous work which performed coarse-grain measurement have provided evidence for the 
existence of opportunities to auto-tune for energy in parallel applications\cite{Tiwari:EnergyAutoTune}.
Using the Resource Centric Reflection daemon tool (RCRtool) developed at RENCI\cite{us}, 
energy 
consumption can be measured at a fine granularity for any OpenMP program. Fine-grain measurements
enable attribution of energy consumption to particular application regions and 
even to the individual lines of codes. This allows accurate study of the correlation between execution time
and energy consumption of an application.

For most scientific applications, nested loops consume the significant 
portion of the total running time. When tuning the application for
better performance and energy usage, some combination of loop optimizations, including loop 
tiling, loop unrolling, and loop fusion, are usually performed on the 
program along with the auto-parallelization.
Determining which set of optimizations produces the best results is hard.
Polyhedral auto-tuning frameworks have shown promising results at simplifying that effort\cite{EJ2011,EJ2012,EJ2013}
for small computation kernels like the Polybench programs\cite{Polybench}. 
In addition to the polybench kernels, we also examine two small applications with the polyhedral framework
to determine the frameworks' effectiveness at reducing overall energy consumption.

This paper has 3 main contributions: 1)  
Fine-grained measurement of execution time and energy consumption with RCRtool
and documenting the correlation between the two. 
2) Additional speedups up to $20\%$ over the already efficient OpenMP baseline 
implementation of a small realistic application using polyhedral optimizations 
on Intel Sandy Bridge and Xeon Phi processors. 3) 
Evaluation of how different architecture effects the utility of the 
polyhedral optimizations techniques for execution time and energy.

The rest of the paper is organized as follows. In Section~\ref{sec:tools}, we describe the tools
used to measure energy consumptions. Section~\ref{sec:benchmarks} describes the benchmarks used for  
measuring the energy consumptions and evaluating the effectiveness of polyhedral 
compilers in optimizing a realistic application. Experimental setup, results and analysis are presented in Section~\ref{sec:setup} and Section~\ref{sec:results}. 
Section~\ref{sec:related} explains and compares with related work. Section~\ref{sec:conclusion} has 
our conclusions.

